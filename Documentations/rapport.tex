%%%%%%%%%%%%%%%%%%%%%%%%%%%%%%%%%%%%%%%%%%%%%%%%%%%%%%%%%%%%%%%%%%%%%%%%%%%%%%%
%     STYLE POUR LES EXPOSES TECHNIQUES 
%         3e ann�e INSA de Rennes
%
%             NE PAS MODIFIER
%%%%%%%%%%%%%%%%%%%%%%%%%%%%%%%%%%%%%%%%%%%%%%%%%%%%%%%%%%%%%%%%%%%%%%%%%%%%%%%

\documentclass[a4paper,11pt]{article}

\usepackage{exptech}       % Fichier (./exptech.sty) contenant les styles pour 
                           % l'expose technique (ne pas le modifier)

%\linespread{1,6}          % Pour une version destin�e � un relecteur,
                           % d�commenter cette commande (double interligne) 
                           
% UTILISEZ SPELL (correcteur orthographique) � acc�s simplifi� depuis XEmacs

%%%%%%%%%%%%%%%%%%%%%%%%%%%%%%%%%%%%%%%%%%%%%%%%%%%%%%%%%%%%%%%%%%%%%%%%%%%%%%%

\title{ \textbf{Rapport\\Distributed Chess Service} }
\markright{Style pour l'expos� technique} 
                           % Pour avoir le titre de l'expose sur chaque page

\author{Paul \textsc{Chaignon}, Thomas \textsc{Fran�ois}, \\
        Damien \textsc{Le Guen}, Cl�ment \textsc{Gautrais}, \\
				Benoit \textsc{Travers} \\
        \\
        Encadreur : Yann \textsc{Ricquebourg}}

\date{}                    % Ne pas modifier
 
%%%%%%%%%%%%%%%%%%%%%%%%%%%%%%%%%%%%%%%%%%%%%%%%%%%%%%%%%%%%%%%%%%%%%%%%%%%%%%%

\begin{document}          

\maketitle                 % G�n�re le titre
\thispagestyle{empty}      % Supprime le num�ro de page sur la 1re page



\begin{abstract}


Le r�sum� est limit� � 10 lignes au maximum.
\end{abstract} 


\section{Introduction}  

\subsection{Attentes du client}
[TODO: Attentes de Loic Helouet.]

\subsection{Sp�cifications}
[TODO: Sp�cifications en bo�te noire.]


\section{Architecture}

\subsection{Interfaces interm�diaires pour les ressources distantes}
[TODO: PHP Wrappers : Appel cURL, parsing et structure.]

\subsection{Utilisation des ressources c�t� serveur}
[TODO: UML, query, version et parsing JSON.]

\subsection{Base de donn�es SQLite}
[TODO: Sch�mas de conception, justification du choix de SQLite.]

\subsection{Parall�lisation des appels aux ressources}
[TODO: Justification du choix des multithreads.]

\subsection{Interface RESTful du serveur central}
[TODO: Interface Jersey : les requ�tes HTTP requises.]


\section{Classement des suggestions de coups}

\subsection{Calcul d'un score}
[TODO: Formule g�n�rale avec somme pour diff�rentes suggestions.]

\subsection{Statistiques}
[TODO: Am�lioration de la formule avec variance et moeynne (centr�e r�duite).]

\subsection{Evolutuion des pond�rations}
[TODO: Evolution automatique des pond�rations.]


\section{Conclusion} 
 



\end{document}

%%%%%%%%%%%%%%%%%%%%%%%%%%%%%%%%%%%%%%%%%%%%%%%%%%%%%%%%%%%%%%%%%%%%%%%%%%%%%%%
