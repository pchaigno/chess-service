\documentclass[a4paper,11pt]{article}

\usepackage[utf8]{inputenc}

\usepackage[french]{babel}

\usepackage{listings}

\title{Distributed Chess Services - Guide technique}

\author{P. Chaignon, C. Gautrais, T. Fran�ois, D. Le Guen, B. Travers}

\date{INSA Rennes - 2012-2013}

\begin{document}

\maketitle

\section{Organisation g�n�rale}

    Le serveur central du projet permet de coordonner les diff�rents appels entre : les requ�tes des clients d'une part, et les recherches de meilleur coup de chaque ressource d'autre part.

\section{Architecture REST}

    Afin d'orchestrer les diff�rents appels entre chaque partie, le projet utilise le framework Grizzly, bas� sur une architecture de type REST.

\section{Gestion des parties de jeux et des ressources}

    L'utilisation de bases de donn�es s'est r�v�l�e n�cessaire sur deux aspects du serveur. Le premier �tant d'enregistrer chaque partie se d�roulant sur le serveur. Ainsi, en plus d'obtenir des informations sur le nombre de parties jou�s et leur d�roulement, il est possible d'�tablir des statistiques pr�cises sur les ressources choisies par le serveur pour donner le meilleur coup.

   

    Les ressources peuvent �tre de trois types : bases de donn�es d'ouverture de partie (les meilleurs d�buts de partie des grands champions d'�checs), bots (�valuant statistiquement le meilleur coup, efficace pour le milieu de partie), et bases de fermeture (meilleurs fin de parties). En interne, chacune poss�de une r�putation servant � l'algorithme de s�lection du coup � renvoyer au client.

\section{Outil graphique de configuration du serveur et de gestion des ressources}

    Afin de faciliter la maintenance du serveur par le maximum de personnes, une interface graphique a �t� mise en place. En plus d'aider � configurer les diff�rents param�tres du serveur (port d'�coute des requ�tes, temps maximum d'attente des r�ponses, nom des bases de donn�es...), cet outil permet de param�trer la liste des ressources disponibles et de modifier leur propri�t�s.

   

    Cet outil utilise la librairie SWT pour l'affichage de l'interface utilisateur.

\end{document}